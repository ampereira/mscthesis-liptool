\chapter*{Abstract}
\label{Abstract}

Most event data analysis tasks in the ATLAS project require both intensive data access and processing, where some tasks are typically I/O bound while others are compute bound. This dissertation work will mainly focus on compute bound issues at the latest stages of the ATLAS detector data analysis, complementing a parallel dissertation work that addresses the I/O bound issues.

The main goal of the work is to design, implement, validate and evaluate an improved and more robust data analysis task which involves tuning the performance of both Top Quark and Higgs Boson reconstruction of events within the framework used for data analysis in ATLAS, to run on (i) homogeneous systems with multiple CPUs and (ii) heterogeneous computing platforms based on multi-core CPU devices coupled to PCI-E boards with many-core devices, such as the Intel Xeon Phi and/or the NVidia Fermi GPU devices.

As a case study, an analysis application will be used, developed by the LIP research group at University of Minho, to tune the Top Quark and Higgs Boson reconstructions, as well as restructure and parallelize other critical areas of this analysis. Two parallelization approaches for homogeneous systems and two for heterogeneous systems are presented and analyzed to identify their limitations, as well as restrictions imposed by the LipMiniAnalysis library, an integral part of every application developed at LIP. An application scheduler is proposed, which purpose is to provide efficient resource usage of multiple CPUs by extracting parallelism of simultaneously executing sequential or also parallel applications when processing large sets of files.

The final goal is use the experience and know-how gained from working with this kind of applications to provide guidelines for efficient resource usage and parallel programming for the LIP researchers. The basis for a tool that automatically extracts parallelism of LIP applications, without the interaction of the programmer, are laid down by the provided the application scheduler and proposed orientations for parallelizing LipMiniAnalysis.