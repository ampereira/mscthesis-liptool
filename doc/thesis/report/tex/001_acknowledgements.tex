\chapter*{Acknowledgments}

Ao meu orientador, professor Alberto Proença, pela orientação exímia que me concedeu, quer construção da tese, quer na escrita da dissertação. Um agradecimento especial por todos os conselhos que deu que ajudaram a moldar tanto o meu rumo académico como pessoal. Ao meu co-orientador, professor António Onofre, pela disponibilidade para colmatar as minhas lacunas na formação na área da Física de Partículas.

Aos meus amigos Pedro Costa e Miguel Palhas, com os quais partilhei muitas discussões e forneceram uma grande ajuda ao meu percurso académico nestes últimos 2 anos. Ao Doutor Nuno Castro e ao meu amigo Juanpe Araque pela disponibilidade que prestaram para atendar a quaisquer dúvidas que me surgissem na Física de Partículas.

Aos professores Alberto Proença, João Sobral e Rui Ralha pela excelente formação em Computação Paralela e Distribuída. A todo o grupo de investigação do LIP Minho por me acolherem e deixarem-me fazer parte deste grande projecto e que, em parceria com a Fundação para a Ciência e Tecnologia (FCT), me concederam uma bolsa de investigação que me permitiu financiar a minha pós-graduação.

Um agradecimento especial à minha família, pais e irmão, pelos muitos sacrifícios que fizeram para me apoiar e permitir chegar onde cheguei. Aos meus amigos mais próximos, que sem eles seria impossível encontrar a motivação e sanidade que tanto precisei. À minha falecida avó Maria da Conceição, que tanto me apoiou para ser médico, dedico esta tese.

\newpage

\vspace*{\fill}
\begin{center}
Em memória da minha falecida avó, Maria da Conceição (1947 - 2005)

\textit{In memory of my late grandmother, Maria da Conceição (1947 - 2005)}
\end{center}
\vspace*{\fill}