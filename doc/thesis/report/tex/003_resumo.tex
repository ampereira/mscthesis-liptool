\chapter*{Resumo}
\label{Resumo}

A maior parte das tarefas de análise de dados de eventos no projeto ATLAS requerem grandes capaci- dades de acesso a dados e processamento, em que a performance de algumas das tarefas são limitadas pela capacidade de I/O e outras pela capacidade de computação. Esta dissertação irá focar-se principalmente nos problemas limitados computacionalmente nas últimas fases de análise de dados do detector do ATLAS, complementando uma dissertação paralela que irá lidar com as tarefas limitadas pelo I/O.

O principal objectivo deste trabalho será desenhar, implementar, validar e avaliar uma tarefa de análise mais robusta e melhorada, que envolve aperfeiçoar a performance das reconstruções dos Top Quarks e bosão de Higgs de eventos dentro da framework usada para a análise de dados no ATLAS, a ser executada em (i) plataformas homogéneas com vários CPUs e (ii) plataformas de computação heterogénea, baseadas em CPUs multicore acoplados a placas PCI-E com dispositivos many-core, tais como o Intel Xeon Phi e/ou os dispositivos GPU NVidia Fermi.

Uma aplicação de análise será usada como caso de estudo, desenvolvida pelo grupo LIP, para melhorar as reconstruções dos Top Quarks e bosão de Higgs, bem como restruturar e paralelizar outras regiões críticas desta análise. Duas abordagens de paralelização para plataformas homogéneas e duas para plataformas heterogéneas serão apresentadas e analisadas, com o objectivo de identificar as suas limitações, bem como as da biblioteca LipMiniAnalysis, uma parte integrante de todas as aplicações desenvolvidas no LIP. É proposto um escalonador de aplicações, cujo objectivo é promover um uso eficiente de recursos em plataformas com vários CPUs através da extracção do paralelismo inerente de executar várias aplicações em simultâneo aquando o processamento de grandes quantidades de dados.

O objectivo final é usar a experiência e \textit{know-how} adquirido ao trabalhar com este tipo de aplicações para establecer um conjunto de directivas para o uso eficiente de recursos e processamento paralelo para os investigadores do LIP. São criadas as fundações para uma ferramenta que automaticamente extraia paralelismo nas aplicações do LIP, sem requerer interacção do programador, através do uso do escalonador de aplicações e de orientações dadas para a paralelização da LipMiniAnalysis.