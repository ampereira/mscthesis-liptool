\chapter*{Resumo}
\label{Resumo}

\section*{Processamento Eficiente de análise de eventos do ATLAS em plataformas homogéneas e heterogéneas}

A maior parte das tarefas de análise de dados de eventos no projeto ATLAS requerem grandes capacidades de acesso a dados e processamento, em que a performance de algumas das tarefas são limitadas pela capacidade de I/O e outras pela capacidade de computação. Esta dissertação irá focar-se principalmente em melhorar a eficiência do código nos problemas limitados computacionalmente nas últimas fases de análise de dados do detector do ATLAS, complementando uma dissertação paralela que irá lidar com as tarefas limitadas pelo I/O.

O principal objectivo deste trabalho será desenhar, implementar, validar e avaliar uma tarefa de análise mais robusta e melhorada, desenvolvida pelo grupo de investigação do LIP na Universidade do Minho. Isto envolve aperfeiçoar a performance das reconstruções do Top Quark e bosão de Higgs de eventos dentro da framework do ATLAS, a ser executada em plataformas homogéneas com vários CPUs e em plataformas de computação heterogénea. A última é baseada em CPUs multicore acoplados a placas PCI-E com dispositivos many-core, tais como o \intel Xeon Phi ou os dispositivos GPU \nvidia Fermi.

Depois de identificar e restructurar as regiões críticas da análise de eventos, duas abordagens de paralelização para plataformas homogéneas e duas para plataformas heterogéneas foram desenvolvidas e avaliadas, com o objectivo de identificar as suas limitações e as restrições impostas pela biblioteca LipMiniAnalysis, uma parte integrante de todas as aplicações desenvolvidas no LIP. Um escalonador de aplicações foi desenvolvido para usar eficientemente os recursos de múltiplos CPUs, através da extracção de paralelismo de execução em simultâneo de tanto aplicações sequenciais como paralelas para processamento de grandes conjuntos de ficheiros de dados.

Um resultado obtido neste trabalho foi um conjunto de directivas para os investigadores do LIP para o uso eficiente de recursos em ambientes paralelos complexos. É possível melhorar as bibliotecas do LIP através do desenvolvimento de uma ferramenta para extrair automaticamente paralelismo das aplicações do LIP, complementado pelo escalonador de aplicações e outras alternativas sugeridas.