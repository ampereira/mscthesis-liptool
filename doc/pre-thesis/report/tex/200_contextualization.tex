
\chapter{Contextualization}
\label{context}

The LHC accelerates two particle beams in opposite directions, causing them to collide at the particle detectors. From this head-on collision between two particles results a limited chain reaction of decaying particles, where the detector records the characteristics of most of the final particles. A schematic representation of the head-on collision, and respective particle decay, is presented in figure \ref{fig:ttbar}, and it is known as the \ttbar system. The detected particles are the bottom quarks (which are detected as a jet of particles) and leptons (electron and muon), while the neutrinos do not react with the detector and, therefore, are not detected. To reconstruct the collision, the characteristics of these neutrinos must be determined. Since this system obeys a set of properties, related to the calibrated model expected from the collision, it is possible to analytically determine the neutrinos characteristics and reconstruct the event (kinematical reconstruction), and then estimate the degree of certainty associated with the computed reconstruction.

\begin{figure}[!htp]
	\begin{center}
		\includegraphics[scale=0.5]{../../common/img/ttbar.png}
		\caption{Schematic representation of the \ttbar system.}
		\label{fig:ttbar}
	\end{center}
\end{figure}

The amount of bottom quark jets and leptons detected can vary between events. However, it is needed at least 2 jets and 2 leptons are needed to reconstruct the \ttbar system, as represented in the figure~\ref{fig:ttbar}, but their amount can reach up to 14. Some of the jets/leptons may not belong to the \ttbar system (as other reactions occur during the collision), so there is a need to choose the ones that most accurately reconstruct the system. Note that the quality of the reconstruction has a strong impact the quality of the research being conducted by the LIP group.

By performing the kinematical reconstruction to each combination of all the bottom quark jets and leptons, two by two, and computing the probability associated with the respective reconstruction, it is possible to chose only the combination that results on the most accurate reconstruction.

Another factor that can affect the accuracy of the reconstruction is the experimental resolution associated with the ATLAS detector. The detected values for the particles (bottom quark jets and leptons) are not fully accurate: the measurements made by ATLAS can have a 2\% fluctuation to the real values. Since these particles are used in the kinematical reconstruction, its accuracy can be impaired. To improve the quality of these reconstructions, the experimental resolution must be compensated. This can be achieved by varying the values of the bottom quark jets and leptons characteristics, such as the mass or momentum, and use them in the kinematical reconstruction. However, this cannot be performed only once: the search space must be covered a certain amount of times to get higher probability of finding a great reconstruction. This means running the kinematical reconstruction as many times as possible, per event, with different variations of the original inputs (jet/lepton combination).

The execution time of the analysis is critical due to the large amounts of data (events) that must be processed. Since for each event it is necessary to reconstruct all the bottom quark jets and leptons combinations, and for each combination a variation is applied several times, the number of kinematical reconstructions per event can quickly rise, increasing the overall time to process an event. A must be achieved balance between the required quality of the reconstruction, directly related to the number of times that the kinematical reconstruction is performed, and the time that takes to process an event. 

The relevance of the kinematical reconstruction (dilep) is even greater in the \tth analysis. This analysis aims to reconstruct the Higgs boson based on the two jets that decay from it. Figure \ref{fig:ttbarhiggs} schematically represents the \ttbar system with the Higgs boson decay and respective jets. However, the jets that decay from the Higgs boson cannot be differentiated from the other similar jets. After performing the \ttbar system reconstruction, i.e., the kinematical reconstruction, and considering the jets used in its best reconstruction, the application uses the remaining jets to reconstruct the Higgs boson. If an event \ttbar system is not properly reconstructed, the Higgs boson reconstruction will not be accurate. The best final reconstruction is the one with the higher combined probabilities of the best kinematical reconstruction and the respective best Higgs boson reconstruction. 

\begin{figure}[!htp]
	\begin{center}
		\includegraphics[scale=0.5]{../../common/img/ttbar_higgs.png}
		\caption{Schematic representation of the \ttbar system with the Higgs boson decay.}
		\label{fig:ttbarhiggs}
	\end{center}
\end{figure}

By increasing the performance of the kinematical reconstruction it is possible to compute more reconstructions per event, leading to better and more accurate results. However, it is not possible to narrow the scope of this dissertation work only to the reconstruction; to get the most efficiency from it, it is necessary to consider other tasks to improve, such as the jet combination, variance appliance and Higgs reconstruction, and eventually re-design the workflow of this section of the application. The LIP research group needs to improve the performance of both kinematical reconstruction and the overall \tth analysis, to improve the overall results quality of the ATLAS project, giving them an advantage over the other teams.

\newpage
